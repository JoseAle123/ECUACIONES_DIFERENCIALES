\documentclass{beamer}
\usetheme{Madrid}
\usepackage[spanish]{babel}
\usepackage[utf8]{inputenc}
\usepackage[T1]{fontenc}
\usepackage{amsmath, amsfonts, amssymb}
\usepackage{graphicx}
\usepackage{hyperref}

\title[Métodos EDO 1\textsuperscript{er} orden]{Métodos de Solución para EDOs de Primer Orden}
\subtitle{Curso: Ecuaciones Diferenciales}
\author{Jose Machaca \and Luis Ramos \and Sebastián Mendoza \and Edilson Mamani \and Giomar Muñoz}
\date{25 de abril de 2025}
\titlegraphic{\includegraphics[height=2.0cm]{unsa.png}}

\begin{document}

\begin{frame}
  \titlepage
\end{frame}

\begin{frame}{Contenido}
  \tableofcontents
\end{frame}

\section{¿Qué es una Ecuación Diferencial?}
\begin{frame}{¿Qué es una Ecuación Diferencial?}
\textit{"Una ecuación diferencial es una ecuación que contiene una función desconocida y una o más de sus derivadas"}~\cite{Zill}.
Estas ecuaciones son fundamentales para modelar fenómenos en física, biología, economía, entre otros.
\end{frame}

\section{Terminología Básica en Ecuaciones Diferenciales}
\begin{frame}{Terminología Básica}
Algunos conceptos clave:
\begin{itemize}
  \item \textbf{Orden}: El orden de la derivada más alta en la ecuación.
  \item \textbf{Linealidad}: Si la función y sus derivadas aparecen linealmente.
  \item \textbf{Solución general y particular}.
\end{itemize}
\end{frame}

\section{Problemas de Valor Inicial (PVI)}
\begin{frame}{Problemas de Valor Inicial}
Un problema de valor inicial consiste en una ecuación diferencial acompañada de una condición inicial: $y(x_0) = y_0$. Su propósito es encontrar una única solución que satisfaga dicha condición.
\end{frame}

\section[Métodos de Solución]{\texorpdfstring{\hyperlink{metodos}{Métodos de Solución para EDOs de Primer Orden}}{}}
\begin{frame}[label=metodos]{Métodos de Solución para EDOs de Primer Orden}
A continuación, se presentan varios métodos para resolver ecuaciones diferenciales ordinarias (EDO) de primer orden:
\begin{itemize}
  \item Separación de variables
  \item Integración directa
  \item Método gráfico (campo direccional)
  \item Factor integrante
  \item Ecuaciones exactas
\end{itemize}
\end{frame}


\section*{Separación de variables}
\begin{frame}{Separación de variables: Introducción}
Ecuaciones diferenciales de primer orden de la forma $y' = f(x)\,g(y)$ se dicen \emph{separables}. Esto significa que la ecuación se puede escribir de forma que todas las expresiones que involucran $y$ queden en un lado de la ecuación y las que involucran $x$ en el otro. La solución se obtiene integrando cada lado.
\end{frame}

\begin{frame}{Separación de variables: Procedimiento}
Para resolver por separación de variables:
\begin{enumerate}
  \item Escribir la ecuación en forma separable: $\displaystyle \frac{dy}{dx} = f(x) g(y)$.
  \item Separar variables: $\displaystyle \frac{dy}{g(y)} = f(x)\,dx$.
  \item Integrar ambos lados: $\displaystyle \int \frac{dy}{g(y)} = \int f(x)\,dx + C$.
  \item Incorporar la constante de integración $C$.
  \item (Opcional) Despejar $y$ en función de $x$ para obtener la solución explícita.
\end{enumerate}
\end{frame}

\begin{frame}{Separación de variables: Ejemplo resuelto}
\textbf{Ejemplo:} Resolver $\displaystyle\frac{dy}{dx} = x\,y$.

Separando variables: $\displaystyle \frac{1}{y}\,dy = x\,dx$. Integrando:
\[
\int \frac{1}{y}\,dy = \int x\,dx \quad\Longrightarrow\quad \ln|y| = \frac{x^2}{2} + C.
\]
Por lo tanto, la solución general es $y = C' e^{x^2/2}$ donde $C' = e^C$.
\end{frame}

\section*{Integración directa}
\begin{frame}{Integración directa: Introducción}
En algunos casos la ecuación diferencial de primer orden se presenta como
\[
\frac{dy}{dx} = f(x),
\]
donde la derivada de $y$ depende solamente de la variable independiente $x$. Este tipo de ecuación se resuelve integrando directamente la función $f(x)$ respecto a $x$.
\end{frame}

\begin{frame}{Integración directa: Procedimiento}
Para resolver por integración directa:
\begin{enumerate}
  \item Verificar que la ecuación está en la forma $\displaystyle \frac{dy}{dx} = f(x)$.
  \item Integrar ambos lados con respecto a $x$: $y(x) = \displaystyle \int f(x)\,dx + C$.
  \item Incluir la constante de integración $C$.
\end{enumerate}
\end{frame}

\begin{frame}{Integración directa: Ejemplo resuelto}
\textbf{Ejemplo:} Resolver $\displaystyle\frac{dy}{dx} = 2x$.

Integramos: $y = \int 2x\,dx + C = x^2 + C$. La solución general es $y = x^2 + C$.
\end{frame}

\section*{Método gráfico (campo direccional)}
\begin{frame}{Método gráfico: Introducción}
El método del campo direccional consiste en dibujar en el plano $(x,y)$ pequeños segmentos de recta cuya pendiente en cada punto $(x,y)$ es $y'=f(x,y)$. Estos segmentos aproximan la pendiente de la solución en ese punto. A partir del campo de pendientes, se puede esbozar la curva solución ajustando una trayectoria que siga los segmentos de manera continua.
\end{frame}

\begin{frame}{Método gráfico: Procedimiento}
Pasos para construir un campo direccional:
\begin{enumerate}
  \item Para una selección de puntos $(x,y)$ en el plano, calcular la pendiente $m = f(x,y)$.
  \item Dibujar en cada punto un pequeño segmento con pendiente $m$.
  \item Para una condición inicial dada $(x_0,y_0)$, trazar una curva que pase por ese punto y siga la dirección de los segmentos dibujados.
\end{enumerate}
Este método permite visualizar las soluciones aunque no siempre proporciona una expresión analítica.
\end{frame}

\begin{frame}{Método gráfico: Ejemplo ilustrativo}
\textbf{Ejemplo ilustrativo:} Considere la ecuación $y' = x - y$. En $(0,0)$ la pendiente es $0-0=0$; en $(1,0)$ es $1-0=1$; en $(0,1)$ es $0-1=-1$, etc. Al dibujar los segmentos correspondientes, se obtiene un campo que orienta las curvas solución. (Aquí no se presenta un cálculo explícito, sino la idea del método).
\end{frame}

\section*{Factor integrante}
\begin{frame}{Factor integrante: Introducción}
Se aplica a ecuaciones diferenciales de primer orden lineales en la forma
\[
\frac{dy}{dx} + P(x)y = Q(x).
\]
El objetivo es encontrar un \emph{factor integrante} $\mu(x)$ tal que al multiplicar la ecuación por $\mu(x)$, ésta se vuelva exacta (o directamente integrable).
\end{frame}

\begin{frame}{Factor integrante: Procedimiento}
Para resolver con factor integrante:
\begin{enumerate}
  \item Identificar $P(x)$ en la ecuación $y' + P(x)y = Q(x)$.
  \item Calcular el factor integrante: $\mu(x) = \exp\left(\int P(x)\,dx\right)$.
  \item Multiplicar toda la ecuación original por $\mu(x)$.
  \item El lado izquierdo se convierte en la derivada de $\mu(x)y$: $\frac{d}{dx}[\mu(x)y] = \mu(x)Q(x)$.
  \item Integrar ambos lados con respecto a $x$: $\mu(x)y = \int \mu(x)Q(x)\,dx + C$.
  \item Despejar $y$ para obtener la solución general.
\end{enumerate}
\end{frame}

\begin{frame}{Factor integrante: Ejemplo resuelto}
\textbf{Ejemplo:} Resolver $\displaystyle\frac{dy}{dx} + y = e^x$.

Aquí $P(x)=1$, así que $\mu(x) = e^{\int 1\,dx} = e^x$. Multiplicando:
\[
e^x \frac{dy}{dx} + e^x y = e^x \cdot e^x \quad\Longrightarrow\quad \frac{d}{dx}(e^x y) = e^{2x}.
\]
Integrando: $e^x y = \int e^{2x} dx + C = \frac{e^{2x}}{2} + C$. Entonces la solución es
\[
y = \frac{e^x}{2} + Ce^{-x}.
\]
\end{frame}

\section*{Ecuaciones exactas}
\begin{frame}{Ecuaciones exactas: Introducción}
Una ecuación de la forma
\[
M(x,y)\,dx + N(x,y)\,dy = 0
\]
se dice \emph{exacta} si existe una función $\Phi(x,y)$ tal que $d\Phi = M\,dx + N\,dy$. Esto ocurre cuando $\frac{\partial M}{\partial y} = \frac{\partial N}{\partial x}$. La solución general se da por $\Phi(x,y) = C$.
\end{frame}

\begin{frame}{Ecuaciones exactas: Procedimiento}
Para resolver una ecuación exacta:
\begin{enumerate}
  \item Verificar si es exacta: comprobar $\partial M/\partial y = \partial N/\partial x$.
  \item Si es exacta, integrar $M(x,y)$ con respecto a $x$: $\Phi(x,y) = \int M(x,y)\,dx + h(y)$.
  \item Derivar $\Phi$ obtenido con respecto a $y$ y compararlo con $N(x,y)$ para encontrar $h'(y)$.
  \item Integrar $h'(y)$ para hallar $h(y)$.
  \item La solución implícita general es $\Phi(x,y) = C$.
\end{enumerate}
\end{frame}

\begin{frame}{Ecuaciones exactas: Ejemplo resuelto}
\textbf{Ejemplo:} Resolver $(2xy + 3)\,dx + (x^2 + 4y)\,dy = 0$.

Aquí $M=2xy+3$ y $N=x^2+4y$. Comprobamos exactitud: $\partial M/\partial y = 2x$, $\partial N/\partial x = 2x$, son iguales. Entonces existe $\Phi$ tal que
\[
\Phi(x,y) = \int (2xy+3)\,dx = x^2y + 3x + g(y).
\]
Derivando con respecto a $y$: $\partial \Phi/\partial y = x^2 + g'(y)$. Debe coincidir con $N = x^2 + 4y$, por lo que $g'(y) = 4y$, de donde $g(y) = 2y^2$. Entonces $\Phi(x,y) = x^2y + 3x + 2y^2$, y la solución implícita es
\[
x^2 y + 2y^2 + 3x = C.
\]
\end{frame}

\section{Ecuaciones Autónomas}
\begin{frame}{Ecuaciones Autónomas}
Una ecuación diferencial es autónoma si la variable independiente no aparece explícitamente, es decir, tiene la forma $y' = f(y)$. Se pueden analizar con diagramas de fases para entender el comportamiento cualitativo de las soluciones.
\end{frame}

\section{Conclusión}
\begin{frame}{Conclusión}
Las ecuaciones diferenciales son una herramienta fundamental en el modelado matemático de fenómenos naturales, tecnológicos y sociales. Desde la descripción de sistemas dinámicos simples hasta la simulación de complejas interacciones físicas, biológicas o económicas, estas ecuaciones permiten capturar la esencia de cómo las variables de un sistema evolucionan con el tiempo o en relación con otras.

De acuerdo con \cite{zill2009ecuaciones}, una ecuación diferencial expresa relaciones entre una función desconocida y sus derivadas, estableciendo una conexión matemática entre cambios y estados. Boyce y DiPrima \cite{boyce2010ecuaciones} enfatizan que las técnicas de solución varían ampliamente dependiendo de la naturaleza de la ecuación, desde métodos analíticos exactos hasta aproximaciones numéricas. Por su parte, Braun \cite{braun1992applications} destaca que muchas ecuaciones no tienen soluciones exactas, lo que subraya la importancia de enfoques cualitativos y gráficos.

Tenenbaum y Pollard \cite{tenenbaum1985ordinary} subrayan que la comprensión de las ecuaciones diferenciales no solo reside en resolverlas, sino en interpretar sus soluciones en contextos reales. Por último, Kreyszig \cite{kreyszig2011advanced} recuerda que su aplicabilidad se extiende más allá de las ciencias básicas, formando la columna vertebral de disciplinas de ingeniería avanzada y matemáticas aplicadas.

En resumen, el estudio de las ecuaciones diferenciales no solo nos permite resolver problemas matemáticos abstractos, sino también conectar dichos problemas con fenómenos tangibles, guiando nuestra comprensión del mundo y ayudando a predecir su comportamiento.
\end{frame}

\section{Bibliografía}
\begin{frame}[allowframebreaks]{Referencias}
\bibliographystyle{apalike}
\bibliography{References}
\end{frame}

\end{document}
